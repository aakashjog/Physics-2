\documentclass[fleqn, a4paper, 12pt, twoside]{article}

\newcounter{recitationcount} %creates a new counter for recitation numbers (must be executed before exsheets is loaded)
\newcommand\recitation{\refstepcounter{recitationcount}}

\usepackage[counter-within = recitationcount]{exsheets}
\usepackage{amsmath, amssymb, amsthm} %standard AMS packages
\usepackage{marginnote} %marginnotes
\usepackage{gensymb} %miscellaneous symbols
\usepackage{commath} %differential symbols
\usepackage{xcolor} %colours
\usepackage{cancel} %cancelling terms
\usepackage{siunitx} %formatting units
\usepackage{tikz, pgfplots} %diagrams
	\usetikzlibrary{calc, hobby, patterns, intersections}
\usepackage{graphicx} %inserting graphics
\usepackage{hyperref} %hyperlinks
\usepackage{datetime} %date and time
\usepackage{ulem} %underline for \emph{}
\usepackage{xfrac} %inline fractions
\usepackage{enumerate} %numbered lists
\usepackage{float} %inserting floats
\usepackage{circuitikz} %circuit diagrams

\newcommand\numberthis{\addtocounter{equation}{1}\tag{\theequation}} %adds numbers to specific equations in non-numbered list of equations

\newcommand{\AxisRotator}[1][rotate=0]{
	\tikz [x=0.25cm,y=0.60cm,line width=.2ex,-stealth,#1] \draw (0,0) arc (-150:150:1 and 1);%
} %rotation symbols on axes

\theoremstyle{definition}
\newtheorem{example}{Example}
\newtheorem{definition}{Definition}

\theoremstyle{theorem}
\newtheorem{theorem}{Theorem}

\newcommand{\curl}{\mathrm{curl\,}}

\makeatletter
\@addtoreset{section}{part} %resets section numbers in new part
\makeatother

\newcommand\blfootnote[1]{%
	\begingroup
	\renewcommand\thefootnote{}\footnote{#1}%
	\addtocounter{footnote}{-1}%
	\endgroup
}

\RenewQuSolPair{question}[name=Recitation \therecitationcount\ -- Exercise]{solution}[name=Recitation \therecitationcount\ -- Solution]

\SetupExSheets{solution/print = true, totoc = true} %prints all solutions by default

%opening
\title{Physics 2 : Recitations}
\author{Aakash Jog}
\date{2014-15}

\begin{document}

\maketitle
%\setlength{\mathindent}{0pt}

\blfootnote
{	
	\begin{figure}[H]
		\includegraphics[height = 12pt]{cc.eps}
		\includegraphics[height = 12pt]{by.eps}
		\includegraphics[height = 12pt]{nc.eps}
		\includegraphics[height = 12pt]{sa.eps}
	\end{figure}
	This work is licensed under the Creative Commons Attribution-NonCommercial-ShareAlike 4.0 International License. To view a copy of this license, visit \url{http://creativecommons.org/licenses/by-nc-sa/4.0/}.
} %CC-BY-NC-SA licencse

\tableofcontents

\newpage
\section{Instructor Information}

\textbf{Dr. Richard Spitzberg}\\
~\\
Office: Ma Aabadot 119\\
E-mail: rms9999@gmail.com\\

\newpage

\part{Electrostatics}

\section{Gravitation and Electromagnetism}

\begin{tabular}{l l}
	Gravitation & Electromagnetism\\
		$F_G = G \dfrac{m_1 m_2}{r^2}$ & $F_E = k \dfrac{q_1 q_2}{r^2}$\\
		$G = 6.7 \times 10^{11} \si{\newton\metre\squared\per\kg\squared}$ & $8.99 \times 10^9 \si{\newton\metre\squared\per\coulomb\squared}$\\
\end{tabular}

\section{Coulomb's Law}

\recitation

\begin{question}
	Four identical charges $q$ are placed in the corners of a square of length $a$. A fifth charge $Q$ is free to move along the straight line perpendicular to the square plane and passing through its centre. When the charge $Q$ is in the same plane as the other charges, all the forces in the system cancel out.
	\begin{enumerate}
		\item Calculate $Q$ for a given $q$ and $a$.
		\item Find the force $\overrightarrow{F(z)}$ acting on the charge $Q$ when it is at height $z$ above the square.
	\end{enumerate}
\end{question}

\begin{solution}[print]
	\begin{figure}[H]
		\begin{tikzpicture}
			\def\a{4};
			
			\draw (-\a/2,-\a/2) rectangle (\a/2,\a/2);
			
			\filldraw ({-\a/2},{-\a/2}) circle [radius = 2pt] node [below left] {$q$};
			\filldraw ({-\a/2},{\a/2}) circle [radius = 2pt] node [above left] {$q$};
			\filldraw ({\a/2},{\a/2}) circle [radius = 2pt] node [above right] {$q$};
			\filldraw ({\a/2},{-\a/2}) circle [radius = 2pt] node [below right] {$q$};
			
			\filldraw (0,0) circle [radius = 2pt] node [left] {$Q$};
		\end{tikzpicture}
	\end{figure}
	Consider $q$ on the top right corner of the square. The total force acting on it is 0. Therefore
	\begin{align*}
		0 &= \dfrac{k q^2}{a^2} \cos \dfrac{\pi}{4} + \dfrac{k q^2}{a^2} \cos \dfrac{\pi}{4} + \dfrac{k q^2}{2 a^2} + \dfrac{2 k Q q}{a^2}\\
		\therefore Q &= - \dfrac{1 + 2 \sqrt{2}}{4} q
	\end{align*}
	~\\
	If $Q$ is at a height $z$ from the plane, the distance between each $q$ and $Q$ is
	\begin{align*}
		r &= \sqrt{z^2 + \left( \dfrac{a}{\sqrt{2}} \right)^2}
	\end{align*}
	Therefore the force of each $q$ on $Q$ is $\dfrac{k Q q}{r^2}$.\\
	Due to symmetry, the components of the forces in the $z$ direction will add up, and all other components will cancel out.\\
	Let the angle between the $z$ direction and the line joining $q$ and $Q$ be $\varphi$.\\
	Therefore, the net force is 
	\begin{align*}
		F &= 4 \dfrac{k Q q}{r^2} \cos \varphi\\
		&= 4 \dfrac{k Q q}{r^2} \dfrac{z}{r}\\
		&= 4 \dfrac{k Q q}{z^2 \left( 1 + \dfrac{a^2}{2 z^2} \right)^{\sfrac{3}{2}}}
	\end{align*}
\end{solution}

\begin{question}
	\begin{enumerate}
		\item A wire of length 3 \si{metre} is charged with 2 \si{\coulomb\per\metre}. What is the wire's total charge?
	\end{enumerate}
\end{question}

\begin{solution}[print]
	\begin{align*}
		\lambda &= \dfrac{Q}{L}\\
		\therefore Q &= L \lambda\\
		&= 6 \si{\coulomb}
	\end{align*}
\end{solution}

\begin{question}
	A wire of length $L$ has the following charge distribution: $\lambda = \lambda_0 \cos \dfrac{\pi x}{L}$, where $x$ is the distance from the wire's edge. What is the wire's total charge?
\end{question}

\begin{solution}[print]
	\begin{align*}
		\lambda &= \dod{q}{x}\\
		\therefore \dod{q}{x} &= \lambda_0 \cos \dfrac{\pi x}{L}\\
		\therefore q &= \int\limits_{0}^{L} \lambda_0 \cos \dfrac{\pi x}{L} \dif x\\
		&= 0
	\end{align*}
\end{solution}

\begin{question}
	A hollow sphere of radius $R$ is uniformly charged with a charge $Q$. Calculate the charge distribution on the surface of the sphere.
\end{question}

\begin{solution}[print]
	\begin{align*}
		\sigma &= \dfrac{Q}{A}\\
		&= \dfrac{Q}{4 \pi R^2}
	\end{align*}
\end{solution}

\begin{question}
	A straight thin wire is uniformly charged with distribution $\lambda$. A charge $q$ is positioned at distance $y_1$ beneath the wire and $r$ away form it.
	\begin{enumerate}
		\item Find the force acting on the charge $q$.
		\item Show that when the charge is positioned in front of the centre of the wire the $\hat{y}$ component of the force is cancelled.
		\item Calculate the force an infinite straight wire will exert on the charge $q$.
	\end{enumerate}
\end{question}

\begin{solution}[print]
	\begin{figure}[H]
		\begin{tikzpicture}
			\def\L{3};
			\def\y{2};
			\def\r{2};
			\def\Y{1};
			
			\def\xMAX{\L + \Y + 1};
			\def\yMAX{\r + 1};
			
			\begin{scope}[gray, -stealth]
				\draw (0,0) -- (\xMAX,0) node [right] {$y$};
				\draw (0,0) -- (0,\yMAX) node [above] {$z$};
			\end{scope}
			
			\draw [ultra thick, red] (0,0) -- (\L,0);
			
			\begin{scope}[|-stealth, yshift = -10]
				\draw (\L,0) -- ++(\Y,0) node [below] {$y_1$};
				\draw (0,0) -- (\y,0) node [below] {$y$};
				\draw [yshift = -10] (0,0) -- (\L,0) node [below] {$L$};
			\end{scope}
			
			\filldraw ({\L + \Y}, \r) circle [radius = 2pt] node [left] {$q$};
			
			\begin{scope}[dashed]
				\draw ({\L + \Y},0) -- ({\L + \Y}, \r) node [midway, right] {$r$};
				\draw (\y,0) -- ({\L + \Y}, \r) node [midway, above left] {$R$};
			\end{scope}
		\end{tikzpicture}
	\end{figure}
	Consider an elemental charge $\dif Q$ of length $\dif y$, at distance $y$ as shown. Let the angle between the line joining $\dif Q$ and $q$ and the $y$ direction be $\theta$.
	\begin{align*}
		F_y &= F \cos \theta\\
		F_z &= F \sin \theta\\
	\end{align*}
	Let
	\begin{align*}
		a &= L + y_1\\
		\therefore R &= \sqrt{r^2 + (a - y)^2}
	\end{align*}
	Therefore,
	\begin{align*}
		\cos \theta &= \dfrac{a - y}{R}\\
		\sin \theta &= \dfrac{r}{R}
	\end{align*}
	Therefore,
	\begin{align*}
		F_y &= k q \int\limits_{0}^{L} \dfrac{\lambda \dif y}{R^2} \dfrac{(a - y)}{R}\\
		&= k q \lambda \int\limits_{0}^{L} \dfrac{\dif y (a - y)}{\left( (a - y)^2 + r^2 \right)^{\sfrac{3}{2}}}\\
		&= k q \lambda \left( \dfrac{1}{\sqrt{{y_1}^2 + r^2}} - \dfrac{1}{\sqrt{a^2 + r^2}} \right)
	\end{align*}
	\begin{align*}
		F_z &= k q \int\limits_{0}^{L} \dfrac{\lambda \dif y}{R^2} \dfrac{r}{R}\\
		&= k q \lambda \int\limits_{0}^{L} \dfrac{r \dif y}{\left( (a - y)^2 + r^2 \right)^{\sfrac{3}{2}}}\\
		&= \dfrac{k q \lambda}{r} \left( \dfrac{a}{\sqrt{r^2 + a^2}} - \dfrac{y_1}{\sqrt{r^2 + {y_1}^2}} \right)
	\end{align*}
	~\\
	When the charge is positioned above the centre of the wire,
	\begin{align*}
		y_1 &= -\dfrac{L}{2}\\
		\therefore a &= \dfrac{L}{2}
	\end{align*}
	Therefore,
	\begin{align*}
		F_y &= k q \lambda \left( \dfrac{1}{\sqrt{{y_1}^2 + r^2}} - \dfrac{1}{\sqrt{a^2 + r^2}} \right)\\
		&= k q \lambda \left( \dfrac{1}{\sqrt{{-\dfrac{L}{2}}^2 + r^2}} - \dfrac{1}{\sqrt{{\dfrac{L}{2}}^2 + r^2}} \right)\\
		&= 0
	\end{align*}
	\begin{align*}
		F_z &= \dfrac{k q \lambda}{r} \left( \dfrac{a}{\sqrt{r^2 + a^2}} - \dfrac{y_1}{\sqrt{r^2 + {y_1}^2}} \right)\\
		&= \dfrac{k q \lambda}{r} \left( \dfrac{\dfrac{L}{2}}{\sqrt{r^2 + {\dfrac{L}{2}}^2}} - \dfrac{-\dfrac{L}{2}}{\sqrt{r^2 + {-\dfrac{L}{2}}^2}} \right)\\
		&= \dfrac{k q \lambda}{r} \left( \dfrac{L}{\sqrt{r^2 + {\dfrac{L}{2}}^2}} \right)\\
		&= \dfrac{k q \lambda}{r} \left( \dfrac{1}{\sqrt{\dfrac{1}{4} + \left( \dfrac{r}{L} \right)^2}} \right)
	\end{align*}
	~\\
	If the line is infinite, $L \to \infty$. Therefore
	\begin{align*}
		F_z &= \dfrac{k q \lambda}{r} \left( \dfrac{1}{\sqrt{\dfrac{1}{4} + \left( \dfrac{r}{L} \right)^2}} \right)\\
		&= \dfrac{2 k q \lambda}{r} 
	\end{align*}
\end{solution}

\section{Gauss' Law}

\recitation
\setcounter{question}{1}

\begin{question}
	A ball of radius $a$ is charged with distribution $\rho$ = $\rho_0 \dfrac{r}{a}$. Find the electric field everywhere.
\end{question}

\begin{solution}
	Consider a spherical Gaussian surface of radius $r$.\\
	If $r \leq a$, the charge in the interior of the Gaussian surface is
	\begin{align*}
		q(r) &= \int\limits_{0}^{r} \dfrac{\rho_0 r}{a} \cdot 4 \pi r^2 \dif r\\
		&= \dfrac{\rho_0}{a} \pi r^4
	\end{align*}
	Therefore, by Gauss' Law,
	\begin{align*}
		E \cdot 4 \pi r^2 &= \dfrac{q(r)}{\varepsilon_0}\\
		\therefore E &= \dfrac{\rho_0 \pi r^4}{4 \pi a r^2}\\
		&= \dfrac{\rho_0 r^2}{4 a \varepsilon_0}
	\end{align*}
	If $r \geq a$, the entire ball of charge is in the interior of the Gaussian surface.\\
	Therefore,
	\begin{align*}
		Q &= q(a)\\
		&= \dfrac{\rho_0}{a} \cdot \pi a^4\\
		&= \rho_0 \pi a^3
	\end{align*}
	Therefore, by Gauss' Law,
	\begin{align*}
		E \cdot 4 \pi r^2 &= \dfrac{Q}{\varepsilon_0}\\
		\therefore E &= \dfrac{Q}{4 \pi r^2 \varepsilon_0}\\
		&= \dfrac{\rho_0 a^3}{4 r^2 \varepsilon_0}
	\end{align*}
	~\\
	Therefore,
	\begin{align*}
		E &=
			\begin{cases}
				\dfrac{\rho_0 r^2}{4 a \varepsilon_0} &;\quad r \leq a\\
				\dfrac{\rho_0 a^3}{4 r^2 \varepsilon} &;\quad r \geq a\\
			\end{cases}
	\end{align*}
\end{solution}

\begin{question}
	An infinitely long cylinder of radius $a$ is charged with distribution $\rho = \rho_0 \dfrac{r}{a}$. Find the electric field everywhere.
\end{question}

\begin{solution}
	Consider a infinite cylindrical Gaussian surface with radius $r$.\\
	If $r \leq a$, the charge in the interior of the Gaussian surface is
	\begin{align*}
		q(r) &= \int\limits_{0}^{r} \dfrac{\rho_0 r}{a} \pi r^2 \dif r\\
		&= \dfrac{2 \pi \rho_0 L r^3}{3 a}
	\end{align*}
	Therefore, by Gauss' Law,
	\begin{align*}
		E \cdot 2 \pi r L &= \dfrac{2 \pi \rho_0 L r^3}{3 a \varepsilon_0}\\
		\therefore E &= \dfrac{\rho_0 r^2}{3 a \varepsilon_0}
	\end{align*}
	If $r \geq a$, the entire cylinder of charge is in the interior of the Gaussian surface.\\
	Therefore,
	\begin{align*}
		Q &= q(a)\\
		&= \dfrac{2 \pi \rho_0 L a^3}{3 a}\\
		&= \dfrac{2 \pi \rho_0 L a^2}{3}
	\end{align*}
	Therefore, by Gauss' Law,
	\begin{align*}
		E \cdot 2 \pi r L &= \dfrac{2 \pi \rho_0 L a^2}{3 \varepsilon_0}\\
		\therefore E &= \dfrac{\rho_0 a^2}{3 \varepsilon_0 r}
	\end{align*}
	~\\
	Therefore,
	\begin{align*}
		E &= 
			\begin{cases}
				\dfrac{\rho_0 r^2}{3 a \varepsilon_0} &;\quad r \leq a\\
				\dfrac{\rho_0 a^2}{3 \varepsilon_0 r} &; \quad r \geq a\\
			\end{cases}
	\end{align*}
\end{solution}

\begin{question}
	Find the electric field due to a thin infinite plane of uniform charge distribution $\sigma$.
\end{question}

\begin{solution}
	Consider a cylindrical Gaussian surface, with ends of area $A$, as shown.
	\begin{figure}[H]
		\begin{tikzpicture}
			\def\h{2};
			\def\r{1};
			\def\L{5};
			
			\draw [blue] ({-\L/2},0) -- ({\L/2},0) node [right] {$\sigma$};
			
			\begin{scope}[red]
				\draw (0,{-\h/2}) circle [x radius = \r, y radius = 0.4*\r];
				\draw (0,{\h/2}) circle [x radius = \r, y radius = 0.4*\r];
				
				\draw ({-\r},{-\h/2}) -- ({-\r},{\h/2});
				\draw ({\r},{-\h/2}) -- ({\r},{\h/2});
			\end{scope}
		\end{tikzpicture}
	\end{figure}
	The charge in the interior of the surface is
	\begin{align*}
		\dif q &= A \sigma
	\end{align*}
	Therefore, by Gauss' Law,
	\begin{align*}
		E_1 \cdot A_1 + E_2 \cdot A_2 &= \dfrac{A \sigma}{\varepsilon_0}\\
		\therefore 2 E A &= \dfrac{A \sigma}{\varepsilon_0}\\
		\therefore E &= \dfrac{\sigma}{2 \varepsilon_0}
	\end{align*}
\end{solution}

\end{document}