\documentclass[fleqn, a4paper, 11pt, oneside]{amsart}
%\usepackage[top = 2cm, bottom = 1cm, left = 1cm, right = 1cm]{geometry}
\usepackage{exsheets, tasks}
\usepackage{amsmath, amssymb, amsthm} %standard AMS packages
\usepackage{marginnote} %marginnotes
\usepackage{gensymb} %miscellaneous symbols
\usepackage{commath} %differential symbols
\usepackage{xcolor} %colours
\usepackage{cancel} %cancelling terms
\usepackage[free-standing-units, space-before-unit]{siunitx} %formatting units
\usepackage{tikz, pgfplots} %diagrams
\usetikzlibrary{calc, hobby, patterns, intersections, decorations.markings}
\usepackage{graphicx} %inserting graphics
\usepackage{hyperref} %hyperlinks
\usepackage{datetime} %date and time
\usepackage{ulem} %underline for \emph{}
\usepackage{xfrac} %inline fractions
\usepackage{enumerate,enumitem} %numbered lists
\usepackage{float} %inserting floats
\usepackage{circuitikz}[american voltages, american currents] %circuit diagrams

\newcommand\numberthis{\addtocounter{equation}{1}\tag{\theequation}} %adds numbers to specific equations in non-numbered list of equations

\newcommand{\AxisRotator}[1][rotate=0]{
	\tikz [x=0.25cm,y=0.60cm,line width=.2ex,-stealth,#1] \draw (0,0) arc (-150:150:1 and 1);%
} %rotation symbols on axes

\theoremstyle{definition}
\newtheorem{example}{Example}
\newtheorem{definition}{Definition}

\theoremstyle{theorem}
\newtheorem{theorem}{Theorem}

\newcommand{\curl}{\mathrm{curl\,}}

\makeatletter
\@addtoreset{section}{part} %resets section numbers in new part
\makeatother

\renewcommand{\thesubsection}{(\arabic{subsection})}
\renewcommand{\thesection}{(\arabic{section})}

%section headings on left
\makeatletter
\def\specialsection{\@startsection{section}{1}%
	\z@{\linespacing\@plus\linespacing}{.5\linespacing}%
	%  {\normalfont\centering}}% DELETED
	{\normalfont}}% NEW
\def\section{\@startsection{section}{1}%
	\z@{.7\linespacing\@plus\linespacing}{.5\linespacing}%
	%  {\normalfont\scshape\centering}}% DELETED
	{\normalfont\scshape}}% NEW
\makeatother

%forces newline after subsection
\makeatletter
\def\subsection{\@startsection{subsection}{3}%
	\z@{.5\linespacing\@plus.7\linespacing}{.1\linespacing}%
	{\normalfont\itshape}}
\makeatother

\settasks{counter-format = tsk[1].}

\SetupExSheets{solution/print = true}

%opening
\title{Physics 2 : Assignment 12}
\author
{
	Aakash Jog\\
	ID : 989323563
}
\date{\formatdate{10}{6}{2015}}

\begin{document}

\tikzset{->-/.style={decoration={
  markings,
  mark=at position #1 with {\arrow{>}}},postaction={decorate}}}

\maketitle
%\setlength{\mathindent}{0pt}

\begin{question}
	A long solenoid of radius $a$ is driven by an alternating current, so that the field inside is sinusoidal $\overrightarrow{B}(t) = B_0 \cos \omega t \hat{z}$.
	A circular loop of wire of radius $\frac{a}{2}$ and resistance $R$ is placed inside the solenoid, coaxial with it.
	Find the current induced in the loop as function of time.
\end{question}

\begin{solution}
	\begin{align*}
		B &= B_0 \cos \omega t\\
	\end{align*}
	Therefore, the flux through the surface bounded by the ring is,
	\begin{align*}
		\Phi_B &= B_0 \cos \omega t \cdot \pi \left( \frac{a}{2} \right)^2\\
		\therefore \dod{\Phi_B}{t} &= -\frac{\pi a^2}{4} B_0 \omega \sin \omega t
	\end{align*}
	Therefore, by Faraday's Law,
	\begin{align*}
		\varepsilon &= -\dod{\Phi_B}{t}\\
		&= \frac{\pi a^2}{4} B_0 \omega \sin \omega t
	\end{align*}
	Therefore,
	\begin{align*}
		I &= \frac{\varepsilon}{R}\\
		&= \frac{\pi a^2}{4 R} B_0 \omega \sin \omega t
	\end{align*}
\end{solution}

\begin{question}
	A long solenoid with radius $a$ and $n$ turns per unit length carries a time-dependent current $I(t)$ in the $\hat{\varphi}$ direction.
	Find the magnitude and direction of the electric field at a distance $s$ from the axis, inside and outside the solenoid.
\end{question}

\begin{solution}
	Consider a virtual Amperian loop of radius $r$ coaxial to the solenoid.\\
	Therefore, by Faraday's Law,\\
	If $s < a$,
	\begin{align*}
		\oint \overrightarrow{E} \cdot \overrightarrow{\dif l} &= -\dod{}{t}\iint \overrightarrow{B} \cdot \dif \overrightarrow{A}\\
		\therefore E \cdot 2 \pi s &= -\dod{}{t}\left( B \pi s^2 \right)\\
		&= -\dod{}{t}\left( \mu_0 n I \pi s^2 \right)\\
		&= -\mu_0 n \pi s^2 \dod{I}{t}\\
		\therefore E &= \frac{\mu_0 n s}{2} \dod{I}{t}
	\end{align*}
	If $r > a$,
	\begin{align*}
		\oint \overrightarrow{E} \cdot \overrightarrow{\dif l} &= -\dod{}{t}\iint \overrightarrow{B} \cdot \dif \overrightarrow{A}\\
		\therefore E \cdot 2 \pi s &= -\dod{}{t}\left( B \pi a^2 \right)\\
		&= -\dod{}{t}\left( \mu_0 n I \pi a^2 \right)\\
		&= -\mu_0 n \pi a^2 \dod{I}{t}\\
		\therefore E &= \frac{\mu_0 n a^2}{2 s} \dod{I}{t}
	\end{align*}
	In both cases, the electric field is directed in the $\hat{\varphi}$ direction.
\end{solution}

\begin{question}
	An alternating current $I = I_0 \cos \omega t$ flows down a long straight wire, and returns along a coaxial conducting tube of radius $a$.
	\begin{enumerate}
		\item
			In which direction does the induced electric field point?
			Radial, circumferential or longitudinal?
		\item Assuming that the field goes to zero as $s \to \infty$, find $\overrightarrow{E}(s,t)$, where $s$ is the distance from the wire.
	\end{enumerate}
\end{question}

\begin{question}
	A long solenoid of radius $a$, carrying $n$ turns per unit length, is looped by a wire with resistance $R$.
	\begin{enumerate}
		\item If the current in the solenoid in increasing at a constant rate $k = \od{I}{t}$, what current flows in the loop, and which way, left or right, does it pass through the resistor?
		\item If the current $I$ in the solenoid is constant but the solenoid is pulled out of the loop and reinserted in the opposite direction, what total charge passes through the resistor?
	\end{enumerate}
\end{question}

\begin{solution}
	\begin{enumerate}[leftmargin = *]
		\item
			\begin{align*}
				\Phi_B &= B \cdot \pi a^2\\
				&= \mu_0 n I \pi a^2\\
				\therefore \dod{\Phi_B}{t} &= \mu_0 n \pi a^2 \dod{I}{t}
			\end{align*}
			Therefore, by Faraday's Law,
			\begin{align*}
				\varepsilon &= -\dod{\Phi_B}{t}\\
				&= -\mu_0 n \pi a^2 \dod{I}{t}\\
				&= -\mu_0 n \pi a^2 k
			\end{align*}
			Therefore,
			\begin{align*}
				I_{\textnormal{resistor}} &= \frac{\varepsilon}{R}\\
				&= -\frac{\mu_0 n \pi a^2 k}{R}
			\end{align*}
			Therefore, the direction of the current in the resistor is opposite to that of the current in the solenoid.
		\item
			If the solenoid is pulled out of the loop and reinserted, the change in the magnetic flux through the loop is,
			\begin{align*}
				\dif \Phi_B &= 2 \pi a^2 \mu_0 n I
			\end{align*}
			Therefore, by Faraday's Law,
			\begin{align*}
				\varepsilon &= -\dod{\Phi_B}{t}\\
			\end{align*}
			Therefore,
			\begin{align*}
				I_{\textnormal{resistor}} &= \frac{\varepsilon}{R}\\
				\therefore \dod{Q}{t} &= -\frac{1}{R} \dod{\Phi_B}{t}\\
				\therefore \dif Q &= -\frac{\dif \Phi_B}{R}\\
				\therefore \dif Q &= -\frac{2 \pi a^2 \mu_0 n I}{R}
			\end{align*}
	\end{enumerate}
\end{solution}

\begin{question}
	A small wire loop of radius $a$ lies at a distance $z$ above the centre of a large loop of radius $b$.
	The planes of the two loops are parallel, and perpendicular to the common axis.
	\begin{enumerate}
		\item Find the flux through the small loop if current $I$ is flowing in the large loop.
		\item Find the flux through the large loop if current $I$ is flowing in the small loop.
		\item Find the mutual inductance of the two loops.
	\end{enumerate}
\end{question}

\begin{solution}
	\begin{enumerate}[leftmargin = *]
		\item
			Assuming $a << b$, the magnetic field passing through the small loop is,
			\begin{align*}
				B &= \frac{\mu_0 I}{2} \frac{b^2}{\left( b^2 + z^2 \right)^{\frac{3}{2}}}
			\end{align*}
			Therefore, the magnetic flux through the small loop is,
			\begin{align*}
				\Phi_B &= B \cdot \pi a^2\\
				&= \frac{\mu_0 I \pi a^2 b^2}{2 \left( b^2 + z^2 \right)^{\frac{3}{2}}}
			\end{align*}
		\item
			The magnetic field passing through a point at distance $r$ from the axis is,
			\begin{align*}
				B &= \frac{\mu_0 I}{4 \pi} \frac{\pi a^2}{r^3} \left( 2 \cos \theta \hat{r} + \sin \theta \hat{\theta} \right)
			\end{align*}
			Therefore, the magnetic flux through the large loop is,
			\begin{align*}
				\varphi_B &= \iint \overrightarrow{B} \cdot \dif \overrightarrow{A}\\
				&= \frac{\mu_0 I \pi a^2 b^2}{2 \left( b^2 + z^2 \right)^{\frac{3}{2}}}
			\end{align*}
		\item
			\begin{equation*}
				{\Phi_B}_1 = {\Phi_B}_2 = M I
			\end{equation*}
			Therefore,
			\begin{align*}
				M &= \frac{\mu_0 \pi a^2 b^2}{2 \left( b^2 + z^2 \right)^{\frac{3}{2}}}
			\end{align*}
	\end{enumerate}
\end{solution}

\end{document}
